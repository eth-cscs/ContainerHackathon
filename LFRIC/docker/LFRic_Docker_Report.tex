\documentclass[twoside,a4paper,12pt]{article}

\usepackage{hyperref,listings,xcolor}
\usepackage{fancyvrb,multirow,caption}
\usepackage[superscript,biblabel]{cite}
\usepackage[margin=2.5cm]{geometry}
\usepackage[affil-it]{authblk}

\hypersetup{
    colorlinks,
    citecolor=purple,
    filecolor=black,
    linkcolor=blue,
    urlcolor=blue
}

\setlength\parindent{0pt}
\setlength{\parskip}{3pt}

\title{Containerisation of LFRic with Docker}
\author{Iva Kav\v ci\v c, Met Office, UK}

\vspace{-8ex}
\date{}

%
%%%%%%%%%%%%%%%%%%%%%%%%%%%%%%%%%%%%%%%%%%%%%%%%%%%%%%%%%%%%%%%%%%%%%%%%%%%%%%%%
%
\begin{document}

\maketitle
%
%%%%%%%%%%%%%%%%%%%%%%%%%%%%%%%%%%%%%%%%%%%%%%%%%%%%%%%%%%%%%%%%%%%%%%%%%%%%%%%%
%
\section{Introduction}\label{introduction}

This report presents work on containerisation of LFRic with Docker during the
ESiWACE2 project
\href{https://www.cscs.ch/events/private-events/event-detail/container-hackathon-for-modellers/}
{``Container Hackathon for Modellers''} in December 2019.
Section~\ref{lfric_overview} gives an overview of LFRic and its main build dependency,
PSyclone. Instructions on building and running LFRic in
\href{https://docs.docker.com/install/}{Docker CE} can be found in
section~\ref{lfric_docker}. Instructions on how to run the LFRic Gungho
container with Sarus on Piz Daint and results of the runs are in
section~\ref{lfric_pizdaint}.

The associated scripts and inputs are hosted in the ``LFRIC'' subdirectory of
the \href{https://github.com/eth-cscs/ContainerHackathon/tree/master}
{Container Hackathon GithHub repository}. For more information on building the
LFRic Docker container please contact
\href{iva.kavcic@metoffice.gov.uk}{Iva Kav\v ci\v c, Met Office} and
\href{https://github.com/lucamar}{Luca Marsella, CSCS} (mentor).
%
%%%%%%%%%%%%%%%%%%%%%%%%%%%%%%%%%%%%%%%%%%%%%%%%%%%%%%%%%%%%%%%%%%%%%%%%%%%%%%%%
%
\section{LFRic and PSyclone:\ A quick overview}\label{lfric_overview}

LFRic~\cite{lfricpaper} is the new weather and climate modelling system being
developed by the UK Met Office to replace the existing Unified Model (UM) in
preparation for exascale computing in the 2020s. LFRic uses the
GungHo~\cite{GungHoDyn} dynamical core and runs on a semi-structured
cubed-sphere mesh.

The design of the supporting infrastructure follows object-oriented principles
to facilitate modularity and the use of external libraries~\cite{MOmodel, MOresnews}.
One of the guiding design principles, imposed to promote performance portability,
is ``separation of concerns'' between the science code and parallel code.
An application called PSyclone, developed at the STFC Hartree Centre, can
generate the parallel code enabling deployment of a single source science code
onto different machine architectures.

PSyclone is a domain-specific compiler and source-to-source translator developed
for use in finite element, finite volume and finite difference codes. Using the
information from a supported API, PSyclone generates code exploiting different
parallel programming models.
%
%%%%%%%%%%%%%%%%%%%%%%%%%%%%%%%%%%%%%%%%
%
\subsection{LFRic repository and wiki}\label{lfric_repo}

The LFRic repository and the associated wiki are hosted at the Met Office
Science Repository Service \href{https://code.metoffice.gov.uk/trac/home}{(MOSRS)}.
The code is BSD-licensed, however browsing the
\href{https://code.metoffice.gov.uk/trac/lfric/wiki}{LFRic wiki} and
\href{https://code.metoffice.gov.uk/trac/lfric/browser}{code repository}
requires login access to MOSRS. Please contact the LFRic team manager,
\href{mailto:steve.mullerworth@metoffice.gov.uk}{Steve Mullerworth},
to be granted access to the repository.

Once the access has been granted the LFRic trunk can be checked out using
\href{https://subversion.apache.org/}{Subversion} or
\href{https://metomi.github.io/fcm/doc/}{FCM} version control systems.
SVN is recommended for checking out the code for runs only as it is easier
to install.
%
%%%%%%%%%%%%%%%%%%%%
%
\subsubsection{LFRic code structure}\label{lfric_code}

The \href{https://code.metoffice.gov.uk/trac/lfric/browser/LFRic/trunk?rev=21509}{LFRic trunk (at revision 21509)}
is structured as follows:

\begin{itemize}
\item \texttt{bin} - Rose executables;
\item \texttt{extra} - Utilities (e.g.\ job submission scripts);
\item \texttt{GPL} - \href{https://github.com/metomi/rose}{Rose} source used in LFRic;
\item \texttt{gungho} - Gungho dynamical core (one of the main science applications);
\item \texttt{infrastructure} - LFRic infrastructure supporting science applications;
\item \texttt{jules} - Interface to the MO \href{https://www.metoffice.gov.uk/research/approach/collaboration/jwcrp/jules}{JULES land surface model};
\item \texttt{lfric\_atm} - LFRic atmospheric model (Gungho dynamical core, UM Physics, JULES and SOCRATES);
\item \texttt{mesh\_tools} - Mesh generation tools;
\item \texttt{miniapps} - ``Standalone'' science and infrastructure applications (e.g. Gravity Wave application);
\item \texttt{socrates} - Interface to the MO radiative transfer (``Suite Of Community RAdiative Transfer codes'') model;
\item \texttt{um\_physics} - Interface to the MO UM Physics parameterisation schemes.
\end{itemize}
%
%%%%%%%%%%%%%%%%%%%%%%%%%%%%%%%%%%%%%%%%
%
\subsection{PSyclone repository and wiki}\label{psyclone_repo}

Both \href{https://github.com/stfc/PSyclone}{PSyclone} and the
\href{https://github.com/stfc/fparser}{Fortran parser} it uses are open source
and hosted on GitHub. Their wikis are also hosted on GitHub:

\begin{itemize}
\item \href{https://github.com/stfc/PSyclone/wiki}{PSyclone wiki};
\item \href{https://github.com/stfc/fparser/wiki}{fparser wiki}.
\end{itemize}

The documentation is hosted on \href{https://readthedocs.org/}{Read the Docs}:
\begin{itemize}
\item \href{https://psyclone.readthedocs.io/en/stable/}{PSyclone documentation};
\item \href{https://fparser.readthedocs.io/en/latest/}{fparser documentation};
\end{itemize}
or PSyclone and fparser repositories for functionality merged to master but not
yet part of an official release.
%
%%%%%%%%%%%%%%%%%%%%
%
\subsubsection{PSyclone in LFRic}\label{psyclone_lfric}

LFRic wiki hosts pages on the use of PSyclone in LFRic, starting with the
\href{https://code.metoffice.gov.uk/trac/lfric/wiki/PSycloneTool}{PSyclone in LFRic wiki}.
As mentioned in section~\ref{lfric_docker}, not every PSyclone release works
with every LFRic trunk revision. The LFRic--PSyclone compatibility table is
given in this
\href{https://code.metoffice.gov.uk/trac/lfric/wiki/LFRicTechnical/VersionsCompatibility}{LFRic wiki (requires login)}.
%
%%%%%%%%%%%%%%%%%%%%%%%%%%%%%%%%%%%%%%%%%%%%%%%%%%%%%%%%%%%%%%%%%%%%%%%%%%%%%%%%
%
\section{Building LFRic Docker container}\label{lfric_docker}
%
%%%%%%%%%%%%%%%%%%%%%%%%%%%%%%%%%%%%%%%%
%
\subsection{Gungho benchmark}\label{gungho_benchmark}

As outlined in section~\ref{lfric_repo}, the LFRic trunk is divided into
several applications. This section outlines how to build a container benchmark
of the Gungho application.

A template Dockerfile to build the LFRic Gungho container is available below.

\begin{Verbatim}[fontsize=\small]
#################################################################################
# LFRic environment: Builds and runs LFRic Gungho benchmark with PSyclone OMP.
# Prerequisites: LFRic build environment built with the system  GCC compiler
                 (version 7.4.0) using Docker template 'lfric_deps.docker' and
                 the relevant installation scripts.
#################################################################################
#
FROM lfric-deps:gnu
#
# Define home and working directories
ENV HOME /usr/local/src
WORKDIR /usr/local/src
#
# Set the compiler environment
ENV COMP_PACKAGE_DIR $HOME/gnu_env
# Set the following environment variables
ENV INSTALL_DIR $COMP_PACKAGE_DIR/usr
ENV BUILD_DIR $COMP_PACKAGE_DIR/build
ENV PFUNIT $INSTALL_DIR
ENV PATH $INSTALL_DIR/bin:$PATH
ENV FC gfortran
ENV FPP "cpp -traditional-cpp"
ENV LDMPI mpif90
ENV FFLAGS "-I$BUILD_DIR/XIOS/inc -I$INSTALL_DIR/include -I$INSTALL_DIR/mod"
ENV LDFLAGS "-L$BUILD_DIR/XIOS/lib -L$INSTALL_DIR/lib -L/usr/lib -lstdc++"
ENV CPPFLAGS "-I$INSTALL_DIR/include -I/usr/include"
ENV LD_LIBRARY_PATH $INSTALL_DIR/lib:$INSTALL_DIR/lib64:$LD_LIBRARY_PATH
# Path to PSyclone configuration file
ENV PSYCLONE_CONFIG /usr/local/share/psyclone/psyclone.cfg
#
# Adds config file for MPICH for Sarus on Piz Daint
RUN echo "/usr/local/src/gnu_env/usr/lib" > /etc/ld.so.conf.d/mpich.conf \
 && ldconfig
#
# For most applications one OMP thread is enough (can be set in batch submit script)
ENV OMP_NUM_THREADS 1
#
# Set option to apply OpenMP optimisations with PSyclone
ENV LFRIC_TARGET_PLATFORM meto-spice
#
# Copy LFRic trunk inside the container
COPY LFRic_trunk.tar .
# Unpack LFRic trunk
RUN tar -xf LFRic_trunk.tar \
# Navigate to "gungho" directory and build application
 && cd LFRic_trunk/gungho \
 && make build -j \
# Navigate to example directory to run the application from
 && cd example
#
# Paths to executables and example directory
ENV PATH $HOME/LFRic_trunk/gungho/bin:$PATH
WORKDIR $HOME/LFRic_trunk/gungho/example
\end{Verbatim}

We have saved the template Dockerfile above as \texttt{lfric\_gungho.docker} and
we built it with the command below:

\begin{Verbatim}[fontsize=\small]
docker build --network=host --add-host $HOSTNAME:127.0.0.1 -f \
  lfric_gungho.docker -t lfric-gungho:gnu .
\end{Verbatim}

The template starts the build from the \texttt{lfric-deps:gnu} Docker container
which contains the libraries needed by the code: MPICH, YAXT, HDF5, NetCDF,
NetCDF-Fortran, NetCDF-C++, XIOS and pFUnit. This dependency container was built
from the script
\href{https://github.com/eth-cscs/ContainerHackathon/blob/master/LFRIC/docker/lfric_deps.docker}
{\texttt{lfric\_deps.docker}} by running

\begin{Verbatim}[fontsize=\small]
docker build --network=host --add-host $HOSTNAME:127.0.0.1 -f \
  lfric_deps.docker -t lfric-deps:gnu .
\end{Verbatim}

The scripts that build the libraries are also provided in the
\href{https://github.com/eth-cscs/ContainerHackathon/tree/master/LFRIC/docker}
{Hackathon LFRic Docker repository}:
\begin{itemize}
\item \href{https://github.com/eth-cscs/ContainerHackathon/blob/master/LFRIC/docker/install_lfric_env.sh}
{\texttt{install\_lfric\_env.sh}} sets up the environment and builds the dependencies of
LFRic without the XIOS;
\item \href{https://github.com/eth-cscs/ContainerHackathon/blob/master/LFRIC/docker/install_xios_env.sh}
{\texttt{install\_xios\_env.sh}} creates architecture files needed to build XIOS and builds XIOS.
\end{itemize}
%
%%%%%%%%%%%%%%%%%%%%%%%%%%%%%%%%%%%%%%%%
%
\subsection{Gravity Wave benchmark}\label{gwave_benchmark}

This section outlines how to build a container benchmark of the Gravity Wave
application, one of the LFRic miniapps (see section~\ref{lfric_repo} for more
details).

A template Dockerfile to build the LFRic Gravity Wave container is available below.

\begin{Verbatim}[fontsize=\small]
#################################################################################
# LFRic environment: Builds and runs LFRic Gravity Wave benchmark.
# Prerequisites: LFRic build environment built with the system  GCC compiler
                 (version 7.4.0) using Docker template 'lfric_deps.docker' and
                 the relevant installation scripts.
#################################################################################
#
FROM lfric-deps:gnu
#
# Define home and working directories
ENV HOME /usr/local/src
WORKDIR /usr/local/src
#
# Set the compiler environment
ENV COMP_PACKAGE_DIR $HOME/gnu_env
# Set the following environment variables
ENV INSTALL_DIR $COMP_PACKAGE_DIR/usr
ENV BUILD_DIR $COMP_PACKAGE_DIR/build
ENV PFUNIT $INSTALL_DIR
ENV PATH $INSTALL_DIR/bin:$PATH
ENV FC gfortran
ENV FPP "cpp -traditional-cpp"
ENV LDMPI mpif90
ENV FFLAGS "-I$BUILD_DIR/XIOS/inc -I$INSTALL_DIR/include -I$INSTALL_DIR/mod"
ENV LDFLAGS "-L$BUILD_DIR/XIOS/lib -L$INSTALL_DIR/lib -L/usr/lib -lstdc++"
ENV CPPFLAGS "-I$INSTALL_DIR/include -I/usr/include"
ENV LD_LIBRARY_PATH $INSTALL_DIR/lib:$INSTALL_DIR/lib64:$LD_LIBRARY_PATH
# Path to PSyclone configuration file
ENV PSYCLONE_CONFIG /usr/local/share/psyclone/psyclone.cfg
#
# Adds config file for MPICH for Sarus on Piz Daint
RUN echo "/usr/local/src/gnu_env/usr/lib" > /etc/ld.so.conf.d/mpich.conf \
 && ldconfig
#
# For most applications one OMP thread is enough (can be set in batch submit script)
ENV OMP_NUM_THREADS 1
#
# Copy LFRic trunk inside the container
COPY LFRic_trunk.tar .
# Unpack LFRic trunk
RUN tar -xf LFRic_trunk.tar \
# Navigate to "gravity_wave" directory and build application
 && cd LFRic_trunk/miniapps/gravity_wave \
 && make build -j \
# Navigate to example directory to run the application from
 && cd example
#
# Paths to executables and example directory
ENV PATH $HOME/LFRic_trunk/miniapps/gravity_wave/bin:$PATH
WORKDIR $HOME/LFRic_trunk/miniapps/gravity_wave/example
\end{Verbatim}

The above template was saved as \texttt{lfric\_gwave.docker} and built with the
command below:

\begin{Verbatim}[fontsize=\small]
docker build --network=host --add-host $HOSTNAME:127.0.0.1 -f \
  lfric_gwave.docker -t lfric-gwave:gnu .
\end{Verbatim}

As for the Gungho benchmark, the template starts the build from the
\texttt{lfric-deps:gnu} Docker container.
%
%%%%%%%%%%%%%%%%%%%%%%%%%%%%%%%%%%%%%%%%
%
\subsection{Library versions and settings}\label{lib_set}

\begin{itemize}

\item All libraries were dynamically linked to make sure that the LFRic container
will use the optimised libraries of the host system:\ the
\href{https://github.com/eth-cscs/ContainerHackathon/blob/master/LFRIC/docker/install_lfric_env.sh}
{\texttt{install\_lfric\_env.sh}} script contains commented out instructions for a static
build if required.

\item The MPICH version used in the current Met Office (MO) LFRic build system
is 3.3. Here we used version 3.1.4 to ensure ABI compatibility with the Cray MPI
library available on Piz Daint: please have a look at the
\href{https://wiki.mpich.org/mpich/index.php/ABI_Compatibility_Initiative}{MPICH Wiki}
for more information on the ABI Compatibility Initiative.

\item HDF5, NetCDF, NetCDF-Fortran, NetCDF-C++ and pFUnit are also slightly
older than in the current MO LFRic build system, whereas
\href{https://www.dkrz.de/redmine/projects/yaxt/}{YAXT} is the same
version. To install the same versions of the above packages that are currently
used by LFRic, please use the alternative install script (dynamic linking
example) \newline
\href{https://github.com/eth-cscs/ContainerHackathon/blob/master/LFRIC/docker/install_lfric_env_current.sh}
{\texttt{install\_lfric\_env\_current.sh}}.

\item \href{https://forge.ipsl.jussieu.fr/ioserver}{XIOS} is a tricky beast to
build as not every revision/release will work with every compiler and/or every
compiler release. For instance, the MO GCC 6.1.0 environment uses revision 1537
which is too old for GCC 7.4.0 used here. Unfortunately, the current XIOS trunk
produces a build which segfaults at runtime for the GCC release and other
libraries used in this container. There is, however, a relatively recent
revision before the segfault bug that was appropriate for this container.

\item Here we used the same PSyclone release (1.7.0) that is used by the current
LFRic trunk (as of 15 December 2019), built with Python 2 environment (the move
to Python 3 and the newest PSyclone 1.8.1 is under way). Not every PSyclone
release will work with every LFRic trunk revision. The LFRic--PSyclone
compatibility table is give in this
\href{https://code.metoffice.gov.uk/trac/lfric/wiki/LFRicTechnical/VersionsCompatibility}
{LFRic wiki (requires login)}.

\end{itemize}
%
%%%%%%%%%%%%%%%%%%%%%%%%%%%%%%%%%%%%%%%%
%
\subsection{Tips \& tricks}\label{tips_tricks}
%
%%%%%%%%%%%%%%%%%%%%
%
\subsubsection{Libraries}\label{libraries_tips_tricks}
%
\begin{itemize}

\item The container tool
\href{https://user.cscs.ch/tools/containers/sarus/}{Sarus} supported on Piz Daint
can add the proper hook to the host MPI library if the command \texttt{ldconfig}
has been run to configure dynamic linker run-time bindings. Since we have
installed the MPICH library in a non-default location, the command \texttt{ldconfig}
would not be able to find the library in the custom path within the container.
Therefore, before running \texttt{ldconfig} we added the path of MPICH to
\texttt{/etc/ld.so.conf.d/mpich.conf} as in the example below (please look
\href{https://unix.stackexchange.com/questions/425251/using-ldconfig-and-ld-so-conf-versus-ld-library-path}
{here} for more information):
\begin{Verbatim}[fontsize=\small]
# Adds config file for MPICH for Sarus on Piz Daint
RUN echo "/usr/local/src/gnu_env/usr/lib" > /etc/ld.so.conf.d/mpich.conf \
 && ldconfig
\end{Verbatim}

\item Some libraries (e.g.\ YAXT) perform a test run of a minimal MPI code
during the configure step of the installation procedure:\ the test might fail
within the Docker container if the local hostname cannot be resolved correctly.
In order to do that, the local hostname should be available in the file
\texttt{/etc/hosts}, which can be achieved adding the option
\texttt{--add-hostname \$HOSTNAME:127.0.0.1} to Docker build command. If this
solution does not work, one needs to edit the \texttt{/etc/hosts} file of the
Docker container directly using \texttt{docker run} and commit the change with
\texttt{docker commit}, since editing the \texttt{/etc/hosts} file is not
possible within the Dockerfile. For more details, please check
\href{https://stackoverflow.com/questions/23112515/mpich2-gethostbyname-failed/23118973}
{this link}.

\item PSyclone configuration file \texttt{psyclone.cfg} can end up in different
locations during PSyclone installation. Please see
\href{https://psyclone.readthedocs.io/en/stable/getting_going.html#configuration}
{PSyclone configuration documentation} for the most common locations. If none of
those work, try the good old \texttt{find / -name "psyclone.cfg"} search.

\end{itemize}
%
%%%%%%%%%%%%%%%%%%%%
%
\subsubsection{LFRic code}\label{lfric_tips_tricks}
%
\begin{itemize}

\item Once the LFRic trunk was checked out, the \texttt{Makefile}s of the tested
Gungho and Gravity Wave applications needed to be modified from
\begin{Verbatim}[fontsize=\small]
export EXTERNAL_DYNAMIC_LIBRARIES = yaxt yaxt_c netcdff netcdf hdf5 \
                                    $(CXX_RUNTIME_LIBRARY)
export EXTERNAL_STATIC_LIBRARIES = xios
\end{Verbatim}
to
\begin{Verbatim}[fontsize=\small]
export EXTERNAL_DYNAMIC_LIBRARIES =
export EXTERNAL_STATIC_LIBRARIES = yaxt yaxt_c xios netcdff netcdf \
                                   hdf5_hl hdf5 z :libstdc++.a
\end{Verbatim}
for the build to complete. For this reason we made changes to \texttt{Makefile}s
outside the container and then copied the tarballs into the container.

\item By default, running command-line \texttt{make build} builds LFRic with MPI
but not OpenMP. To enable PSyclone OpenMP optimisations, the environment
variable \texttt{LFRIC\_TARGET\_PLATFORM} was set to \texttt{meto-spice} value for
the Gungho benchmark.

\end{itemize}
%
%%%%%%%%%%%%%%%%%%%%%%%%%%%%%%%%%%%%%%%%%%%%%%%%%%%%%%%%%%%%%%%%%%%%%%%%%%%%%%%%
%
\section{Running LFRic Docker container on Piz Daint}\label{lfric_pizdaint}
%
%%%%%%%%%%%%%%%%%%%%%%%%%%%%%%%%%%%%%%%%
%
\subsection{Gungho benchmark}\label{gungho_pizdaint}

The Docker image of the LFRic Gungho benchmark was loaded with \texttt{sarus}
and run on Piz Daint. The Slurm batch script below was used as a template for
running the benchmark.

\begin{Verbatim}[fontsize=\small]
#!/bin/bash -l
#SBATCH --job-name=job_name
#SBATCH --time=01:00:00
#SBATCH --nodes=1
#SBATCH --ntasks-per-node=6
#SBATCH --cpus-per-task=1
#SBATCH --constraint=gpu
#SBATCH --output=lfric-gungho.%j.out
#SBATCH --error=lfric-gungho.%j.err

module load daint-gpu
module load sarus
module unload xalt
srun sarus run --mount=type=bind,source=$PWD/input/gungho,\
     destination=/usr/local/src/LFRic_trunk/gungho/example \
     --mpi load/library/lfric-gungho:gnu \
     bash -c "export OMP_NUM_THREADS=$SLURM_CPUS_PER_TASK && \
     gungho ./gungho_configuration.nml
\end{Verbatim}

The local folder \texttt{input/gungho} is a copy of the folder
\texttt{LFRIC\_trunk/gungho/example} and contains:

\begin{itemize}

\item Namelist \texttt{gungho\_configuration.nml};

\item Four mesh files required for the multigrid preconditioner
(\texttt{l\_multigrid} flag set to \texttt{.true.}\ in the \texttt{\&multigrid}
namelist) used in Gungho (\texttt{mesh24.nc}, \texttt{mesh12.nc},
\texttt{mesh6.nc} and \texttt{mesh3.nc});

\item \texttt{iodef.xml} file required for parallel IO.

\end{itemize}

The IO file, however, was not used as the \texttt{use\_xios\_io} flag in the
\texttt{\&io} namelist was set to \texttt{.false.}. If no diagnostic output
(e.g.\ plots) are required, it is recommended that the \texttt{write\_diag} flag
in the \texttt{\&io} namelist is also set to \texttt{.false.}. All input
files are available as the
\href{https://github.com/eth-cscs/ContainerHackathon/blob/master/LFRIC/docker/input-gungho.tar.gz}
{Gungho input archive} in the
\href{https://github.com/eth-cscs/ContainerHackathon/tree/master/LFRIC/docker}
{Hackathon LFRic Docker repository}. The mesh files and related namelists
required for the Gungho higher resolution tests below (\texttt{C48},
\texttt{C96} and \texttt{C192} mesh configurations) were produced as described
in Appendix~\ref{mesh_generation}.

The Gungho application produces a single text output for a serial run (single
MPI task) or \texttt{PET**.gungho.Log} outputs for multiple MPI tasks, as well
as checksums stored in \texttt{gungho-checksum.txt} output. Depending on the
\texttt{\&io} namelist's settings, it also produces diagnostic outputs such as

\begin{itemize}

\item Overall application run times stored in \texttt{timer.txt} (set by the
\texttt{subroutine\_timers} flag to \texttt{.true.});

\item Overall application counter of halo calls stored in
\texttt{halo\_calls\_counter.txt} (set by the \texttt{subroutine\_counters} flag
to \texttt{.true.});

\item Plots of the results (set by the \texttt{write\_diag} and
\texttt{diagnostic\_frequency} options);

\item XIOS client output and error logs if \texttt{use\_xios\_io} flag is set to
\texttt{.true.}\ (\texttt{xios\_client.out} and \texttt{xios\_client.err} for a
serial run/single MPI task; \texttt{xios\_client\_**.out} and
\texttt{xios\_client\_**.err} for multiple MPI tasks).

\end{itemize}
%
%%%%%%%%%%%%%%%%%%%%
%
\subsubsection{Results}\label{gungho_pizdaint_results}

Tables~\ref{gungho_table_1}-\ref{gungho_table_3} show times for completing
the Gungho benchmark on Piz Daint Cray XC50 with different mesh resolutions,
number of nodes, MPI tasks and OpenMP threads. Table~\ref{gungho_table_4} shows
results of the additional tests for the \texttt{C24} mesh configuration,
run with 1 MPI task and on 1, 2 and 4 OpenMP threads, respectively.

The times are from the LFRic \texttt{timer.txt} outputs and Slurm job outputs.
Note that the Slurm completion times are longer, as they include overheads of
job submission (prologue and epilogue).

\begin{table}[!h]
\begin{tabular}{ |c|c|c|c|c| }
\hline
\multirow{3}{*}{OMP threads} & \multicolumn{4}{ |c| }{C24, 1 node} \\ \cline{2-5}
           & \multicolumn{2}{ |c| }{LFRic timer}  &\multicolumn{2}{ |c| }{Slurm time} \\ \cline{2-5}
           & 1 MPI p.n. & 6 MPI p.n. & 1 MPI p.n. & 6 MPI p.n. \\
\hline
1          & 328.91 s   & 72.22 s    & 00:08:06 (486 s) & 00:01:56 (116 s) \\
\hline
2          & 185.56 s   & 44.33 s    & 00:03:24 (204 s) & 00:00:57 (57 s) \\
\hline
\end{tabular}
\caption{Gungho benchmark runtimes on Piz Daint for \texttt{C24} mesh
configuration, run on 1 compute node with 1 and 6 MPI tasks per node,
respectively.}\label{gungho_table_1}
\end{table}

\begin{table}[!h]
\begin{tabular}{ |c|c|c|c|c| }
\hline
\multirow{3}{*}{OMP threads} & \multicolumn{4}{ |c| }{C48, 1 node} \\ \cline{2-5}
           & \multicolumn{2}{ |c| }{LFRic timer}  &\multicolumn{2}{ |c| }{Slurm time} \\ \cline{2-5}
           & 1 MPI p.n. & 6 MPI p.n. & 1 MPI p.n. & 6 MPI p.n. \\
\hline
1          & 1354.05 s  & 280.76 s   & 00:22:51 (1371 s) & 00:05:00 (300 s) \\
\hline
2          & 776.13 s   & 177.2 s    & 00:13:18 (798 s)  & 00:03:26 (206 s) \\
\hline
\end{tabular}
\caption{Gungho benchmark runtimes on Piz Daint for \texttt{C48} mesh
configuration, run on 1 compute node with 1 and 6 MPI tasks per node,
respectively.} \label{gungho_table_2}
\end{table}

\begin{table}[!h]
\begin{tabular}{ |c|c|c|c|c| }
\hline
\multirow{3}{*}{OMP threads} & \multicolumn{2}{ |c| }{C96, 6 nodes, 6 MPI p.n.}
                             & \multicolumn{2}{ |c| }{C192, 6 nodes, 6 MPI p.n.} \\ \cline{2-5}
           &  LFRic timer  &  Slurm time        &  LFRic timer  &  Slurm time     \\ \cline{2-5}
\hline
1          &  194.59 s     &  00:03:33 (213 s)  &  801.52 s     &  00:13:40 (820 s)  \\
\hline
2          &  128.43 s     &  00:02:57 (177 s)  &  517.95 s     &  00:08:57 (537 s)  \\
\hline
\end{tabular}
\caption{Gungho benchmark runtimes on Piz Daint for \texttt{C96} and
\texttt{C192} mesh configurations, respectively, run on 6 compute nodes with 6
MPI tasks per node.} \label{gungho_table_3}
\end{table}

\begin{table}[!h]
\begin{tabular}{ |c|c|c| }
\hline
\multirow{2}{*}{OMP threads} & \multicolumn{2}{ |c| }{C24, 1 node, 1 MPI p.n.} \\ \cline{2-3}
           &  LFRic timer    &  Slurm time       \\
\hline
1          &  394.76 s       &  00:06:58 (418 s) \\
\hline
2          &  231.69 s 	     &  00:04:32 (272 s)  \\
\hline
4          &  153.93 s 	     &  00:03:14 (194 s)  \\
\hline
\end{tabular}
\caption{Gungho benchmark runtimes on Piz Daint for \texttt{C24} mesh
configuration, run on 1 compute node with 1 MPI task per node and 1, 2 and
4 OpenMP threads, respectively.} \label{gungho_table_4}
\end{table}

%
%%%%%%%%%%%%%%%%%%%%%%%%%%%%%%%%%%%%%%%%
%
\subsection{Gravity Wave benchmark}\label{gwave_pizdaint}

The Docker image of the LFRic Gravity Wave benchmark was loaded with
\texttt{sarus} and run on Piz Daint. The Slurm batch script below was used as a
template for running the benchmark.

\begin{Verbatim}[fontsize=\small]
#!/bin/bash -l
#SBATCH --job-name=lfric-gwave
#SBATCH --time=00:30:00
#SBATCH --nodes=1
#SBATCH --ntasks-per-node=1
#SBATCH --cpus-per-task=1
#SBATCH --partition=normal
#SBATCH --constraint=gpu

module load daint-gpu
module load sarus
module unload xalt
srun sarus run \
     --mount=type=bind,source=$PWD/input/gwave,\
     destination=/usr/local/src/LFRic_trunk/gwave/example \
     --mpi load/library/lfric-gwave:gnu \
     bash -c "export OMP_NUM_THREADS=$SLURM_CPUS_PER_TASK && \
     gravity_wave ./gravity_wave_configuration.nml"
 \end{Verbatim}

The local folder \texttt{input/gwave} contains the namelist
\texttt{gravity\_wave\_configuration.nml} and the mesh file \texttt{mesh24.nc}:\
both files are available in the
\href{https://github.com/eth-cscs/ContainerHackathon/blob/master/LFRIC/docker/input-gwave.tar.gz}
{Gravity Wave input archive} of the
\href{https://github.com/eth-cscs/ContainerHackathon/tree/master/LFRIC/docker}
{Hackathon LFRic Docker repository}.
This folder is a trimmed-down copy of the folder
\texttt{LFRIC\_trunk/miniapps/gravity\_wave/example}. There are also other input
files present in the LFRic repository:\ \texttt{mesh12.nc}, \texttt{mesh6.nc} and
\texttt{mesh3.nc} (for the multigrid preconditioner) and \texttt{iodef.xml}
(for parallel IO). They are not present in the \texttt{input/gwave} archive as
neither \texttt{l\_multigrid} nor \texttt{use\_xios\_io} flags were set to
\texttt{.true.}\ for the Piz Daint run.

The Gravity Wave application produces outputs similar to the Gungho application
depending on the namelists' settings. On a single Piz Daint Cray XC50 node the
benchmark took around 5 minutes to complete on a single MPI task (truncated
output below).

\begin{Verbatim}[fontsize=\small]
Batch Job Summary Report for Job "lfric-gwave" (18507334) on daint
---------------------------------------------------------------
Start                End                  Elapsed     Timelimit
-------------------  -------------------  ----------  ---------
2019-12-04T12:06:38  2019-12-04T12:11:50  00:05:12    01:00:00
---------------------------------------------------------------
Username    Account     Partition   NNodes   Energy
----------  ----------  ----------  ------  --------------
hck01       hck         normal           1   29.18K joules
\end{Verbatim}

%
\section{Summary}\label{summary}

The LFRic Docker containers for Gungho and Gravity Wave benchmark were built
inside the Ubuntu 18.04 virtual environment on a laptop and then deployed and
run on Piz Daint using the local libraries. Building containers was somewhat
challenging due to the LFRic build dependencies and quirks of the Ubuntu virtual
machine. A Docker container for only the libraries and build dependencies needed
by the code seems as a very useful tool, for both building benchmark containers
and providing LFRic software stack to collaborators who may not have access to
the requirements or expertise to build the environment.

Running the Docker containers with Sarus on Piz Daint showed to be relatively
straightforward in comparison with the process of building them. The Gungho
benchmark was successfully run with different mesh resolutions, number of nodes,
MPI tasks and OpenMP threads. The results show good scaling, with the actual
runtimes shorter than the Slurm completion times as expected.

The Met Office has been exploring various approaches for containerisation
of its models and the experiences from this hackathon will be very useful in
coming up with the optimal strategies for different applications.
%
%%%%%%%%%%%%%%%%%%%%%%%%%%%%%%%%%%%%%%%%%%%%%%%%%%%%%%%%%%%%%%%%%%%%%%%%%%%%%%%%
%
\begin{thebibliography}{unsrt}
%
\bibitem{lfricpaper} Adams SV, Ford RW, Hambley M, Hobson JM, Kavcic I, Maynard
CM, Melvin T, Mueller EH, Mullerworth S, Porter AR, Rezny M, Shipway BJ and
Wong R (2019):\ \href{https://doi.org/10.1016/j.jpdc.2019.02.007}
{\em LFRic:\ Meeting the challenges of scalability and performance portability
in Weather and Climate models.} Journal of Parallel and Distributed Computing.
%
\bibitem{GungHoDyn} \href{https://www.metoffice.gov.uk/research/foundation/dynamics/next-generation}
{\em GungHo - a next generation atmospheric dynamical core for weather and
climate modelling.} Accessed:\ 15 December 2019.
%
\bibitem{MOmodel} \href{https://www.metoffice.gov.uk/research/approach/modelling-systems/lfric}
{\em LFRic - a modelling system fit for future computers.}
Accessed:\ 15 December 2019.
%
\bibitem{MOresnews} \href{https://www.metoffice.gov.uk/research/news/2019/gungho-and-lfric}
{\em Next generation atmospheric model development.} Accessed:\ 15 December 2019.
\end{thebibliography}
%
%%%%%%%%%%%%%%%%%%%%%%%%%%%%%%%%%%%%%%%%%%%%%%%%%%%%%%%%%%%%%%%%%%%%%%%%%%%%%%%%
%
\appendix

\section{LFRic mesh generation}\label{mesh_generation}

This section describes how to generate higher resolution cubed-sphere meshes
used in the LFRic Gungho runs on Piz Daint (see section~\ref{gungho_pizdaint}
for more details on the runs). To generate the required meshes, we used the
\texttt{lfric-gungho:gnu} Docker container (see section~\ref{lfric_docker}
for the container build instructions).
%
%%%%%%%%%%%%%%%%%%%%%%%%%%%%%%%%%%%%%%%%
%
\subsection{LFRic mesh tools}\label{lfric_mesh_tools}

LFRic application \texttt{mesh\_tools} (see section~\ref{lfric_repo} for more
details on LFRic code structure) contains all prerequisites for the generation
of cubed-sphere and biperiodic Cartesian meshes (we only used the former in the
Piz Daint tests).

Running \texttt{make build} or \texttt{make build -j} in the \texttt{mesh\_tools}
directory produces two executables for mesh generation in the \texttt{bin/}
directory, \texttt{cubedsphere\_mesh\_generator} and
\texttt{planar\_mesh\_generator}, and the executable \texttt{summarise\_ugrid}
that summarises mesh files information (LFRic stores meshes in UGRID format).

Cubed-sphere or planar meshes can then be produced by running
\begin{Verbatim}[fontsize=\small]
../bin cubedsphere_mesh_generator cubedsphere.nml
\end{Verbatim}
or
\begin{Verbatim}[fontsize=\small]
../bin planar_mesh_generator planar_mesh.nml
\end{Verbatim}
from the \texttt{mesh\_tools/example} directory. The \texttt{C48} mesh can be
generated by copying the \texttt{cubedsphere.nml} to
\texttt{cubedsphere\_C48.nml} namelist, modifying the new namelist as shown below
\begin{Verbatim}[fontsize=\small]
&cubedsphere_mesh_generator
  mesh_filename = 'mesh_C48.nc'
  nmeshes       = 1
  mesh_names    = 'dynamics'
  edge_cells    = 48
  smooth_passes = 0
/
\end{Verbatim}
and then running
\begin{Verbatim}[fontsize=\small]
../bin cubedsphere_mesh_generator cubedsphere_C48.nml
\end{Verbatim}
from the \texttt{mesh\_tools/example} directory. The \texttt{C96} and
\texttt{C192} meshes are also produced in a similar way.
%
%%%%%%%%%%%%%%%%%%%%%%%%%%%%%%%%%%%%%%%%
%
\subsection{Using LFRic container for mesh generation}\label{lfric_container_mesh}

To build the mesh generation executables, we navigated to the
\texttt{LFRic\_trunk/mesh\_tools} directory in the \texttt{lfric-gungho:gnu}
container and modified the \texttt{Makefile} from
\begin{Verbatim}[fontsize=\small]
export EXTERNAL_DYNAMIC_LIBRARIES = netcdff netcdf hdf5 yaxt yaxt_c
\end{Verbatim}
to
\begin{Verbatim}[fontsize=\small]
export EXTERNAL_DYNAMIC_LIBRARIES =
export EXTERNAL_STATIC_LIBRARIES = netcdff netcdf hdf5 yaxt yaxt_c
\end{Verbatim}
before running \texttt{make build -j} (similar \texttt{Makefile} modifications
were also required for the Gungho and Gravity Wave benchmarks;\ see
section~\ref{lfric_tips_tricks} for details). The required mesh files,
\texttt{C48}, \texttt{C96} and \texttt{C192}, were then generated as described
above.
%
%%%%%%%%%%%%%%%%%%%%%%%%%%%%%%%%%%%%%%%%
%
\subsection{Higher resolution Gungho benchmark on Piz Daint}\label{lfric_pizdaint_high}

The generated mesh files were copied from the LFRic Gungho container to the host
Ubuntu 18.04 virtual environment platform in order to be transferred to
Piz Daint for the runs described in section~\ref{lfric_pizdaint}.

Files can be copied from a Docker container to the host platform by using command
\begin{Verbatim}[fontsize=\small]
docker cp <containerId>:/file/path/within/container /host/path/target
\end{Verbatim}
where \texttt{docker ps} gives the container ID.

Running Gungho with multigrid preconditioner requires four-component mesh chain:\
base mesh and three lower-resolution meshes, each one twice as coarse as the
mesh before. Hence, running \texttt{C48} jobs required mesh files for the
\texttt{C48} base mesh and \texttt{C24}, \texttt{C12} and \texttt{C6} coarser
resolution meshes. It also required modifying the \texttt{gungho\_namelist.nml}
input file as shown below.

\begin{itemize}

\item \texttt{filename} option in the \texttt{\&base\_mesh} namelist was modified from
\begin{Verbatim}[fontsize=\small]
filename        = 'mesh_C24.nc'
\end{Verbatim}
to
\begin{Verbatim}[fontsize=\small]
filename        = 'mesh_C48.nc'
\end{Verbatim}

\item \texttt{ugrid} option in the \texttt{\&multigrid} namelist was modified from
\begin{Verbatim}[fontsize=\small]
ugrid      = 'prime', 'mesh_C12.nc', 'mesh_C6.nc', 'mesh_C3.nc'
\end{Verbatim}
to
\begin{Verbatim}[fontsize=\small]
ugrid      = 'prime', 'mesh_C24.nc', 'mesh_C12.nc', 'mesh_C6.nc'
\end{Verbatim}
For easier work the modified namelist input file was saved as \newline
\texttt{gungho\_configuration\_C48.nml}. The higher resolution (\texttt{C96} and
\texttt{C192}) mesh chains and namelist input files were generated in a similar
way and then transferred to Piz Daint.
%
%%%%%%%%%%%%%%%%%%%%%%%%%%%%%%%%%%%%%%%%
%
\subsection{Higher resolution Gungho benchmark inside the container}\label{lfric_container_high}

The generated mesh files can be copied into the \texttt{LFRic\_trunk/gungho/example}
directory inside the LFRic Gungho container. After modifying the namelist input
files as shown above, simply run e.g.\ \texttt{C48} configuration as
\begin{Verbatim}[fontsize=\small]
../bin/gungho gungho_configuration_C48.nml
\end{Verbatim}
in serial or
\begin{Verbatim}[fontsize=\small]
mpirun -np $MPI_TASKS ../bin/gungho gungho_configuration_C48.nml
\end{Verbatim}
for parallel runs (it is recommended to have multiples of 6 \texttt{MPI\_TASKS}
for the cubed-sphere runs). The container changes need to be committed so that
the generated meshes and namelist files are saved for later, either in the same
or in a new container.

\end{itemize}

\end{document}
